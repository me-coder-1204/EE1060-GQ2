\section{Significance of time-shift}
\begin{enumerate}
\item In time-delayed systems, a positive shift ($t_0$) means that the system responds later to changes in the input. This can cause the system to react too slowly, which may lead to oscillations or instabilities in feedback systems, especially if the system relies on feedback loops (like in control systems or biological systems).
\item A negative shift ($t<0$) might result in the system anticipating the input and responding before it occurs. This could cause overreaction or instability if the system is not designed to handle anticipatory behavior.
\item In devices like thermostat, the system may respond to temperature changes with a delay because of the time it takes to heat/cool the room. By shifting of kernel, if the delay becomes too large, then the system might not be able to track the input accurately, leading to instability (or) oscillations.
\item In the context of signal processing, shifting the kernel corresponds to a phase shift in the signal, e.g., in delay filters, the impulse response can be delayed.
\item In biological systems, time delays are common in processes like hormonal responses, neural signals, or metabolic pathways. Shifting the kernel models how these systems react over time. For example, the delayed release of insulin after glucose intake could be modeled using a time-shifted kernel.

\end{enumerate}
